\documentclass{article}

\usepackage[margin=1.5in]{geometry}
\usepackage{amsmath}
\usepackage{tabu}
\usepackage{url,graphicx}
\usepackage{fancyhdr}
\usepackage{enumitem}

\title{CSCI 338 Project 1}
\author{Wade Mouer}
\date{13 September 2019}

\pagestyle{fancy}
\lhead{CSCI338\\Project 1: Introduction to GitHub}
\rhead{Fox, Mouer, Robinson, Sharpe, Spradlin\\13 September 2019}

\linespread{1.3}
\begin{document}

\section*{Introduction to GitHub}
\subsection*{Preliminaries}
The members of this group are Zachary Fox, Wade Mouer, Stephen Robinson, Justin Sharpe, and Caleb Spradlin. In this project, the group had two main tasks to perform. First, each group member was to complete the Introduction to GitHub Learning Lab. This lab was meant to familiarize group members with the GitHub platform and walk the group members through the basic operations they will need to use to collaborate with others on GitHub. Second, the group put together a simple web page with contributions from each member. This web page was built with HTML and CSS, with images added from a folder in the repository. 
\subsection*{Learning Lab}
The first component of this project was the Learning Lab on GitHub. Group members had varying levels of experience using GitHub, but agreed that it was a useful tool for guiding someone through the basic flow of GitHub. Some members had used GitHub in previous classes, others had used GitHub for individual programming or retrieving open source material, and others were complete beginners. For those who were not used to GitHub, it provided a quick crash course to get them up to speed, and for those who were more familiar, it provided a refresher and a reminder of GitHub's features. The learning lab did not present any significant obstacles for the group. 
\subsection*{GitHub Use}
For the most part, collaboration using GitHub worked smoothly for this project and the conflicts that arose were not significant road blocks to the group. There was slight confusion early on due to group members' unfamiliarity using GitHub. Some work was written over on the master branch rather than on a separate branch, which simply meant some code had to be written twice -- little harm done. There was one merge conflict that the group experienced during this project. Spradlin had made alterations on his branch and encountered a merge conflict when he merged his branch. This merge conflict was rather trivial and anticlimactic, as it had to do mostly with spacing and numbering of lines and did not result in a screaming match between programmers. The conflict was easily resolved, as Spradlin's changes were chosen over the existing branch. This was more of a conflict between programmer versus GitHub rather than programmer versus programmer. That being said, the merge conflict was a bit of a surprise, as the conflicts between Spradlin's branch and the Master branch were so minute that he expected GitHub to be able to merge them more smoothly. 
\subsection*{Robot Friend}
The web page created displays a robotic figure supplemented with images gathered by the group members from the internet. These images include the head of a llama making up the head of the robotic figure. The torso is made up of an image of Marvel's The Incredible Hulk, and the robot's left arm, fittingly, is a mechanized arm and hand. The end result is a simple web page, complete with a title header reading "Robot Friend" and list of the names of the group members at the bottom. 
\subsection*{Reflections}
As to be expected, the main challenges of this project had to do with the group members getting more comfortable with GitHub rather than with nitty-gritty HTML and CSS coding. For members of the group who had used GitHub before, this project provided a different context: instead of using it as a solo programmer or using it to access material made public by someone else, GitHub was used as a collaboration platform for a small team, which required the use and finesse of more unfamiliar features. For others, GitHub was entirely (or almost entirely) foreign, so learning the ropes and understanding the project flow was daunting at first. 

This also led to slight issues and frustrations getting the hang of GitHub. As mentioned above, some code had to be rewritten after being altered on the master branch instead of a separate branch. Several group members also encountered an issue where they made changes to their branch and thought they had successfully merged it, but the changes weren't showing up when they returned to look at it. That said, GitHub also eased collaboration by organizing the code and establishing procedures for changes to be made. It also made it easy to keep up on the code that the other team members were working on, and it helped communication through included messages and the README. 

This project gave the opportunity to encounter those frustrations and work through them with a relatively low pressure project. Using GitHub on the command line was another skill that several group members made effort to practice, and this project helped with that, increasing comfort using the command line.

The group members agreed that this project provided valuable practice using GitHub -- partly because it is a useful tool for collaboration among a small team, but also because it is a widely used platform in the professional world, with a significant portion of job listings requiring or preferring those with GitHub experience. 


\end{document}
